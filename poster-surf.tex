\documentclass[final,hyperref={pdfpagelabels=false}]{beamer}
\usepackage{grffile}
\mode<presentation>{\usetheme{ECOMP}}
\usepackage[portuges, brazil]{babel}   
\usepackage[utf8]{inputenc}
\usepackage{amsmath,amsthm, amssymb, latexsym}
%\usepackage{times}\usefonttheme{professionalfonts}  % obsolete
%\usefonttheme[onlymath]{serif}
\boldmath
\usepackage[orientation=portrait,size=a0,scale=1.4,debug]{beamerposter}
% change list indention level
% \setdefaultleftmargin{3em}{}{}{}{}{}


%\usepackage{snapshot} % will write a .dep file with all dependencies, allows for easy bundling

\usepackage{array,booktabs,tabularx}
\newcolumntype{Z}{>{\centering\arraybackslash}X} % centered tabularx columns
\newcommand{\pphantom}{\textcolor{ta3aluminium}} % phantom introduces a vertical space in p formatted table columns??!!

\listfiles

%%%%%%%%%%%%%%%%%%%%%%%%%%%%%%%%%%%%%%%%%%%%%%%%%%%%%%%%%%%%%%%%%%%%%%%%%%%%%%%%%%%%%%
\graphicspath{{figures/}}
 
\title{\huge Desenvolvimento de um poster para o E-Comp, com os logos que tem geometrias diferentes}
\author{Alessandro Camillo Gimenez de Menezes\inst{1} Ezequiel França dos Santos\inst{1}, Gabriel Vieira Figueiredo Tomaz\inst{1}}
\institute[SENAC]{Centro Universitário Senac - Campus Santo Amaro}
\date[Dezembro 3th, 2014]{Dezembro 03, 2014}

%%%%%%%%%%%%%%%%%%%%%%%%%%%%%%%%%%%%%%%%%%%%%%%%%%%%%%%%%%%%%%%%%%%%%%%%
\newlength{\columnheight}
\setlength{\columnheight}{105cm}
%%%%%%%%%%%%%%%%%%%%%%%%%%%%%%%%%%%%%%%%%%%%%%%%%%%%%%%%%%%%%%%%%%%%%%%%
\begin{document}%Inicia o documento
\begin{frame}
  \begin{columns}%Inicia as Colunas
    % ---------------------------------------------------------%
    % Configurar a primeira coluna
\begin{column}{.49\textwidth}
      \begin{beamercolorbox}[center,wd=\textwidth]{postercolumn}
        \begin{minipage}[T]{.95\textwidth} % tweaks the width, makes a new \textwidth
          \parbox[t][\columnheight]{\textwidth}{ % must be some better way to set the the height, width and textwidth simultaneously
            % Since all columns are the same length, it is all nice and tidy.  You have to get the height empirically
            % ---------------------------------------------------------%
            % fill each column with content
            
            %%RESUMO%%%%%%%%%%%%%%%%%
            \begin{block}{Resumo}
                 Este trabalho apresenta o projeto de um web crawler, que obtém dados do Instituto de Economia Agrícola do estado de São Paulo e propoem algumas visualizações destes dados, baseando-se em características da cesta básica, apresentando um design de informação lúdico para o usuário final..
            \end{block}
            \vfill
            %%INTRODUÇÃO%%%%%%%%%%%%%
            \begin{block}{Introdução}
             Atualmente a conexão dos usuários a internet, seja através de computadores, tablets, smartphones, relógios ou quaisquer outra interfaces digitais evoluíram e se tornaram muito mais acessíveis aos usuários. Junto com esta evolução, o mercado está sempre desenvolvendo meios de trazer aos usuários acesso ao grande volume de dados já existentes na Word Wide Web novos formatos, com especial atenção ao design de informação. Neste trabalho apresentamos um web crawler que obtém dados do Instituto de Economia Agricola do Estado de São Paulo, e com estes dados apresentamos algumas possibilidades de visualizações, seguindo algumas tendências de visualizações lúdicas com apelo ao design de informação, apresentando algumas propostas e possibilidades de visualizações, utilizando fatores como correlações diretas e indiretas e médias de comportamento no tempo.              
            \end{block}
            \vfill
            %%%%%%%%%%OBJETIVOS%%%%%%%%%%%%%%
            \begin{block}{Objetivos}
             Este trabalho apresenta a proposta de um sistema de visualização de dados, seguindo uma arquitetura de web crawler - web service - visualização. 
             
Neste trabalho, para o obtenção dos dados foi utilizada uma técnica de crawling 
		
            \end{block}
        	\vfill    
            %%%%%%%%%%%%%METODOLOGIA%%%%%%%%
            \begin{block}{Metodologia}
              \begin{itemize}
              \item Grid-based local feature extraction instead of interest points
              \item Local descriptors:
                \begin{itemize}
                \item upright descriptor versions achieved better results
                \item SURF-128 better than SURF-64
                \end{itemize}
              \item System robustness: manually aligned/unaligned/partially occluded faces
                \begin{itemize}
                \item SURF more robust to illumination
                \item SIFT more robust to changes in viewing conditions
                \end{itemize}
              \item RANSAC-based system combination and outlier removal
              \end{itemize}
              Lorem Ipsum é simplesmente uma simulação de texto da indústria tipográfica e de impressos, e vem sendo utilizado desde o século XVI, quando um impressor desconhecido pegou uma bandeja de tipos e os embaralhou para fazer um livro de modelos de tipos. Lorem Ipsum sobreviveu não só a cinco séculos, como também ao salto para a editoração eletrônica, permanecendo essencialmente inalterado. Se popularizou na década de 60, quando a Letraset lançou decalques contendo passagens de Lorem Ipsum, e mais recentemente quando passou a ser integrado a softwares de editoração eletrônica como Aldus PageMaker.
            \end{block}
            \vfill
          }
          % ---------------------------------------------------------%
          % end the column
        \end{minipage}
      \end{beamercolorbox}
    \end{column}
    % ---------------------------------------------------------%
    % Fim da coluna 1

    % ---------------------------------------------------------%
    % Ajuste da Coluna 2
    \begin{column}{.49\textwidth}
      \begin{beamercolorbox}[center,wd=\textwidth]{postercolumn}
        \begin{minipage}[T]{.95\textwidth} % tweaks the width, makes a new \textwidth
          \parbox[t][\columnheight]{\textwidth}{ 
            \begin{block}{Databases}
              \begin{columns}
                \begin{column}{.6\textwidth}
                  \begin{itemize}
                  \item AR-Face 
                    \begin{itemize}
                    \item variations in illumination
                    \item many different facial expressions
                    \end{itemize}
                  \item CMU-PIE
                    \begin{itemize}
                    \item variations in illumination (frontal images from the illumination subset)
                    \end{itemize}
                  \end{itemize}
                \end{column}
                \begin{column}{.39\textwidth}
                  \includegraphics[width=0.22\linewidth]{hanselmann-databases/arface/train/png/occlusions/arneutral/m-016-1}
                  \-
                  \includegraphics[width=0.22\linewidth]{hanselmann-databases/arface/train/png/cropped/m-016-4}
                  \-
                  \includegraphics[width=0.22\linewidth]{hanselmann-databases/arface/test/png/occlusions/ar1sun/m-016-8}
                  \-
                  \includegraphics[width=0.22\linewidth]{hanselmann-databases/arface/test/png/occlusions/ar1scarf/m-016-11}
                  \vskip3ex

                  \includegraphics[width=0.22\linewidth]{hanselmann-databases/cmupie/test/png/cropped/00-27-02}
                  \-
                  \includegraphics[width=0.22\linewidth]{hanselmann-databases/cmupie/test/png/cropped/00-27-17}
                  \-
                  \includegraphics[width=0.22\linewidth]{hanselmann-databases/cmupie/test/png/cropped/00-27-18}
                  \-
                  \includegraphics[width=0.22\linewidth]{hanselmann-databases/cmupie/test/png/cropped/00-27-20}
                \end{column}
              \end{columns}
            \end{block}
            \vfill
            
            %%%%
            \begin{block}{Results: Manually Aligned Faces}
              \begin{itemize}
              \item AR-Face: 110 classes, 770 train, 770 test
              \end{itemize}
              \vskip-0.5ex
              \begin{table}
                \centering
                \small
                \begin{tabular}{@{} p{.2\linewidth} p{.18\linewidth} p{.25\linewidth} r r r @{}}
                  \toprule 
                  Descriptor &  Extraction        & \multicolumn{1}{p{.2\linewidth}}{\# Features}           & \multicolumn{3}{c @{}}{Error Rates [\%]}         \\
                  \cmidrule(l){4-6}
                             &                    &                                                      & \multicolumn{1}{c}{Maximum} & \multicolumn{1}{c}{Grid} & \multicolumn{1}{c @{}}{Grid-Best} \\
                  \cmidrule(r){1-1}  \cmidrule(lr){2-2}   \cmidrule(lr){3-3}                               \cmidrule(lr){4-4}            \cmidrule(lr){5-5}         \cmidrule(l){6-6}  
                  SURF-64    & IPs                & \pphantom{1}64 $\times$ 5.6 (avg.)                    & 80.64     &  84.15  & 84.15     \\
                  SIFT       & IPs                & 128 $\times$ 633.78 (avg.)                           & 1.03      &  95.84  & 95.84      \\
                  \addlinespace
                  SURF-64    & 64x64-2 grid       & \pphantom{1}64 $\times$ 1024                          & 0.90      &  0.51   & 0.90      \\  
                  SURF-128   & 64x64-2 grid       & 128 $\times$ 1024                                    & 0.90      &  0.51   & 0.38       \\
                  SIFT       & 64x64-2 grid       & 128 $\times$ 1024                                    & 11.03     &  0.90   & 0.64       \\
                  \addlinespace
                  U-SURF-64  & 64x64-2 grid       & \pphantom{1}64 $\times$ 1024                          & 0.90      &  1.03   & 0.64      \\ 
                  U-SURF-128 & 64x64-2 grid       & 128 $\times$ 1024                                    & 1.55      &  1.29   & 1.03       \\ 
                  U-SIFT     & 64x64-2 grid       & 128 $\times$ 1024                                    & \textbf{0.25}  &  \textbf{0.25}   & \textbf{0.25}  \\
              
                  \bottomrule
                \end{tabular}
              \end{table} 
            \end{block}
            \vfill
            

            \begin{block}{Conclusão}
              O trabalho apresenta como principal contribuição, uma possibilidade de reconhecimento de padrões pré-determinados  para controle de jogos utilizando métodos simples, porém com resultados, dentro de seus limites, precisos.

Os algoritmos propostos são de fácil implementação e não requerem uma abordagem matemática profunda para sua compreensão e aplicação. O trabalho mostra ainda, que estes métodos, com pouca modificação poderiam ser utilizados em qualquer outro tipo de interface por visão computacional, uma vez que seus algoritmos possuem complexidades relativamente médias.
%%% References
            \end{block}
            \vfill
            \begin{block}{Referencias Bibliograficas}
              \begin{itemize}
              \item Eu nao me preocupei com acentos e etc, logo, verifique 
              \item Misturar paçoquita cremosa com Nutella é bem legal
              \item LEMBRE-SE DO VFILL, SE VAI USAR OU NAO EM CADA BLOCO.
              \end{itemize}
            \end{block}
            \vfill
          }
           
          % ---------------------------------------------------------%
          % end the column
        \end{minipage}
      \end{beamercolorbox}
    \end{column}
    % ---------------------------------------------------------%
    % end the column
  \end{columns}
  \vskip1ex
\end{frame}
\end{document}


%%%%%%%%%%%%%%%%%%%%%%%%%%%%%%%%%%%%%%%%%%%%%%%%%%%%%%%%%%%%%%%%%%%%%%%%